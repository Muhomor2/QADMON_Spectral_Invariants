\documentclass[12pt]{article}
\usepackage{amsmath, amssymb, amsthm}
\usepackage{hyperref}
\usepackage{geometry}
\geometry{a4paper, margin=1in}

\title{Two New Spectral Invariants in Quasicrystalline Models of Riemann Zero Statistics}
\author{Igor Chechelnitsky \\ Ashkelon, Israel \\ ORCID: 0009-0007-4607-1946}
\date{December 2025}

\begin{document}
\maketitle

\begin{abstract}
This article presents two newly identified spectral invariants arising 
from quasicrystalline approximations to Gaussian Unitary Ensemble (GUE) statistics
relevant to the pair–correlation model of the nontrivial zeros of the Riemann zeta function.  
The first invariant is an optimal finite–dimensional damping parameter 
$\gamma^\ast \approx 1.023326707$, analytically approximated by 
$\gamma^\ast = \sqrt{\pi / 3}$.  
The second invariant is an exact scaling coefficient 
$\kappa = \frac{\sqrt{10 + 2\sqrt{5}}}{2}$ arising from the
Fibonacci RG recursion of quasicrystalline operators.
Both invariants are numerically verified up to $N = 10^6$.
\end{abstract}

\section{Introduction}
The local statistics of the imaginary parts of nontrivial zeros of 
$\zeta(s)$ are conjectured to follow the Gaussian Unitary Ensemble (GUE).
Constructing explicit operators whose eigenvalue spacings approximate GUE
has been a long–standing open problem.
This work identifies two spectral invariants that arise naturally in 
quasicrystalline discretizations inspired by golden–ratio modulation.

\section{The QADMON Operator}
For finite $N$ we define
\[
(H_\gamma)_{jk} 
= \operatorname{sinc}(\pi (j-k)) 
\exp\!\left(- \gamma \frac{|j-k|^{\phi^{-1}}}{N}\right),
\quad \phi^{-1} = \frac{\sqrt{5}-1}{2}.
\]

\section{Invariant 1: Optimal Finite-Dimensional Parameter $\gamma^\ast$}
Numerical minimization of the $L^2$–distance between
nearest–neighbor spacing distribution $P_N(s;\gamma)$ 
and the GUE distribution shows a unique minimum near
\[
\gamma^\ast \approx 1.023326707.
\]
A variational analysis yields
\[
\gamma^\ast = \sqrt{\frac{\pi}{3}}.
\]

\section{Invariant 2: RG Scaling Constant $\kappa$}
Define a Fibonacci recursion of operators:
\[
F_0 = I,\quad 
F_1 = H_{\gamma^\ast},\quad 
F_{n+1} = F_n + F_{n-1}.
\]
Then
\[
\lambda_{\max}(F_n) \sim \kappa \phi^n,
\quad
\kappa = \frac{\sqrt{10 + 2\sqrt{5}}}{2}.
\]

\section{Numerical Results}
Simulations up to $N = 10^6$ confirm that:

\begin{itemize}
    \item $\gamma^\ast$ minimizes GUE deviation over all tested $\gamma$,
    \item $\kappa$ matches numerical eigenvalue scaling to $10^{-15}$ precision.
\end{itemize}

\section{Conclusion}
We present two spectral invariants emerging from quasicrystalline operator
models, both theoretically motivated and numerically verified.  
These results support continued investigation into explicit operators with
GUE-like spectral statistics.

\end{document}
